日本の農業機械発達しについて

戦後、どのようにして日本の農業機械は捉えられ、発達してきたのか、文献より言葉を引用しつつ、その歴史を紐解く。

『牽引車工學』
京都帝国大学教授 農学博士 田村豊
京都帝国大学講師 農学士 増田正三

をデジタル化プロジェクト。

\section{序}
大東亜戦争は彼我の精神力の争闘であるとともに、正に科学技術力の争覇戦である。
わが日本は緒戦に於て幾多の新兵器を駆使し的をして顔色をなからしめた。
真珠湾攻撃に於ける特殊潜航艇の奇襲、馬らいお気に於ける空らいの威力、馬らい沖の於ける会場起動
